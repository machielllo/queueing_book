% arara: pdflatex: { shell: yes, interaction: nonstopmode }
% arara: pythontex: {verbose: yes, rerun: modified }
% arara: pdflatex: { shell: yes, interaction: nonstopmode }
% arara: clean: { extensions: [ aux, blg, idx, ilg, ind, log, out, pytxcode, rel, toc ] }
% !arara: clean: { files: [ ans.tex, hint.tex] }
% arara: pdflatex
% arara: clean: { extensions: [ aux, blg, idx, ilg, ind, log, out, pytxcode, rel, toc ] }
% !arara: clean: { files: [ ans.tex, hint.tex] }


\documentclass[queueing-book-solution.tex]{subfiles}
\externaldocument{queueing-book}

\opt{solutionfiles,check}{
\loadgeometry{tufte}
\Opensolutionfile{hint}
\Opensolutionfile{ans}
}

\begin{document}

\section{Kendall's Notation}
\label{sec:kendalls-notation}

As became apparent in~\cref{sec:constr-discr-time,sec:constr-gg1-queu}, the construction of a single-station queueing process involves three main elements: the distribution of job inter-arrival times and the service times, and the number of servers present to process jobs.



To characterize the type of queueing process it is common to use
\recall{Kendall's abbreviation} $A/B/c/K$, where $A$ is the common distribution of the
iid inter-arrival times, $B$ the common distribution of the iid service times, $c$ the
number of servers, and $K$ the system size, i.e., the total number of customers that can be simultaneously present, whether in queue or in service.\sidenote{The meaning of $K$ differs among authors. Sometimes it stands for
 the capacity of the queue, not the entire system. In this book $K$ corresponds to the system's size.}
In this notation it is assumed that jobs are served in FIFO order, and, in this book, we only consider the FIFO service discipline.


Two inter-arrival and service distributions are the most important in queueing theory: the exponential distribution denoted with the shorthand $M$, as it is memoryless, and a general distribution (with the implicit assumption that its first moment is finite) denoted with $G$. We write $D$ for a deterministic (constant) random variable.

\newthought{A few important examples} are the following queueing processes: $M/M/1$, $M/G/1$ and $G/G/c$.
A model that is often used to determine the number of beds needed in (a ward of) a hospital is the $M/M/c/c$ queue.
The motivation is as follows.
Practice tells us that patient inter-arrival times are memoryless, hence exponentially distributed.
Data of patients treatment times shows that these times are also well-described by an exponential distribution.
Next, there are $c$ beds available, and each bed can serve one patient. When all beds are occupied, the hospital is `full'.


\newthought{When at an arrival} a number of jobs arrive simultaneously (like a bus at a restaurant), we say that a batch arrives.
Likewise, the server can work in batches, for instance, when an oven processes multiple jobs at the same time.
We write $A^X/B^Y/c$ to indicate that we consider batch arrivals and batch services.
When $X\equiv 1$ or $Y \equiv 1$, i.e., single batch arrivals or single batch services, we suppress the $X$ or $Y$ in the queueing formula.

\input{trailer}

%%% Local Variables:
%%% mode: latex
%%% TeX-master: t
%%% End:
