% arara: clean: { extensions: [ aux, blg, idx, ilg, ind, log, out, pytxcode, rel, toc ] }
% arara: pdflatex: { shell: yes, interaction: nonstopmode }
% arara: pythontex: {verbose: yes, rerun: modified }
% arara: pdflatex: { shell: yes, interaction: nonstopmode }
% arara: clean: { extensions: [ aux, blg, idx, ilg, ind, log, out, pytxcode, rel, toc ] }
% !arara: clean: { files: [ ans.tex, hint.tex] }

% arara: pdflatex
% arara: clean: { extensions: [ aux, blg, idx, ilg, ind, log, out, pytxcode, rel, toc ] }
% !arara: clean: { files: [ ans.tex, hint.tex] }


\documentclass[queueing-book-solution.tex]{subfiles}
\externaldocument{queueing-book}

\opt{solutionfiles,check}{
\loadgeometry{tufte}
\Opensolutionfile{hint}
\Opensolutionfile{ans}
}

\begin{document}

\section{Setups and Batch Processing}
\label{sec:setups-batch-proc}

In some cases, machines have to be setup before they can start producing items.
Consider, for instance, a machine that paints red and blue items.\sidenote{In realistic problems there can be hundreds of families.}
When the machine requires a color change, it may be necessary to clean up the machine, which takes time.
Another example is an oven that needs a temperature change when different item types require different production temperatures.
Service operations form another setting with setup times: when servers (personnel) have to move from one part of a building to another, the time spent moving cannot be spent on serving customers.\sidenote{Often, the setup time depends on the sequence in which the families are produced, for example, a switch in color from white to black takes less cleaning time and cost than from black to white.
  The problem then becomes to determine first a production sequence that minimizes the sum of the setup times to produce the families, and then find suitable batch sizes to minimize the average waiting times.
  This problem becomes yet more realistic (and very hard) when jobs have due-dates too.
}


In all such cases, the setups consume a significant amount of time; in fact, setup times of an hour or longer are not uncommon.
Clearly, in such situations, it is necessary to produce in batches: a server processes a batch of jobs of one type or at one location, then  changes from type or location, starts serving a batch of another type or at another location, and so on.

Here we focus on the effect of change-over, or setup, times on the average sojourn time of jobs.
First we make a model and provide a list of elements required to compute the expected sojourn time of an item, then we illustrate how to use these elements in a concrete case.

\newthought{Assume there are} two job families, e.g., red and blue, each served by the same single server.
Jobs arrive at rate $\lambda_r$ and $\lambda_b$. The SCV of job inter-arrival times is given by $C_a^2$.
Jobs of both type require the same average \recall{net processing time} of $\E{S_0}$, provided the server is already setup for the correct job color, and have variance $\V{S_0}$.
Setup times are iid and  have mean $\E R$ and  variance $\V R$, and are assumed to be independent of job service times.


A job's sojourn time is built up as follows.
First, an arrival is assembled with other jobs of the same color to form a batch of constant size $B$.
For simplicity, $B$ is the same for both colors.\sidenote{In general, $B$ should depend on the arrival rate of the job type. For instance, when red jobs arrive much faster than blue jobs, it takes much longer to fill a blue batch than a red batch when batch sizes are equal.}
Once a batch is complete, the batch enters a queue (of batches).
After some time the batch reaches the head of the queue, and is ready to move to the machine as soon as it becomes available.
To serve a batch, the machine first performs a setup, and then processes each job individually until the batch is finished.
Once a job's service is completed, it may, or may not\sidenote{Depending on the type of production.}, have to wait for the other jobs in the same batch before it can leave.
Let us make a model of this.


\newthought{When a job} of type $i$ arrives, the expected time for its batch to be formed is given by\sidenote{~\cref{ex:48}}
\begin{equation}\label{eq:79}
 \E{\W_i} = \frac{B-1}{2\lambda_i}, \quad i\in \{r, b\}.
\end{equation}
When the batch is complete, it joins the queue, so we next compute the average time in queue.\sidenote{Note that we shift interpretation: batches (not individual items) form a queue and move as a whole to the server.}

\newthought{Now that we} have a batch of jobs, we need to estimate the average time a batch spends in queue.
For this we can use Sakasegawa's formula;  we only have to find expressions for each of its components.

The total arrival rate of jobs is $\lambda= \lambda_b+\lambda_r$. Thus, $\lambda_B=\lambda/B$ is the arrival rate of \emph{batches}.
It is easy to see that the expected service time of a batch is
\begin{equation}\label{eq:90}
\E{S_B} = \E R + B\E{S_0}.
\end{equation}
Sometimes it useful to convert the effects of the  setup time into an \recall{effective processing time} of an individual job:
\begin{equation}\label{eq:60}
  \E{S} = \frac{\E{S_B}}{B} = \frac{\E{R}} B +  \E{S_0}.
 \end{equation}
With the batch arrival rate and expected batch service time, the load becomes  $\rho = \lambda_B \E{S_B}$

It is essential that $B$ is sufficiently large to ensure that $\rho<1$.
With~\cref{eq:90} this leads to the condition that $B$ must be larger than some minimal batch size, i.e.,
\begin{equation*}
B> B_m = \frac{\lambda \E R}{1-\lambda \E{S_0}}.
\end{equation*}

Now that we have identified the service time of a batch and the load, we only need the squared coefficients of variation for the batch inter-arrival times $C_{a,B}^2$ and the batch service times\sidenote{Once again, for the \emph{batches}, not the jobs!} $C_{s,B}^2$ for Sakasegawa's formula.
We find that\sidenote{~\cref{ex:490} \cref{ex:491}}
 \begin{align}\label{eq:82}
C_{a,B}^2 &= \frac{C_{a}^2}B, &
C_{s, B}^2 &= \frac{\V{S_B}}{(\E{S_B})^2},
\end{align}
where
\begin{equation*}
  \V{S_B} = \V R  + B \V{S_0}.
\end{equation*}
By dividing by $B$ we see that the variance of the effective processing time is
\begin{equation}  \label{eq:107}
  \V{S} %= \frac{\V R}{B} + \V{S_0}
  = \V R / B + \V{S_{0}}.
\end{equation}
Indeed, the variance of the effective service times is larger than the variance of the net processing times.

\newthought{It is left} to find a rule to determine what happens to an item after it has been processed.
If the job has to wait until all jobs in the batch are served, the expected time it spends at the server is $\E R + B \E{S_0}$.
However, if the item can leave
right after being served, the expected time at the server is\sidenote{\cref{ex:492}}
\begin{equation}\label{eq:85}
\E{R} + \frac{B-1}{2}\E{S_0} + \E{S_0}= \E{R} + \frac{B+1}{2}\E{S_0};
\end{equation}
the first component is the average time a job has to wait before its service starts, the second is its service time.
Our model is complete!\sidenote{The model in~\cref{ex:19} is more general as it allows the server to work at varying rates.}



\newthought{We can obtain a number} of important insights from the above model.
Using~\cref{ex:103} we plot the sojourn time $\E{J_r}$ of a red job for various values of $B$ in the figure at the right.


\begin{marginfigure}
\begin{pycode}[batch]
import matplotlib.pyplot as plt
import tikzplotlib
from matplotlib import style
style.use('ggplot')


labda = 3  # per hour
ES0 = 15.0 / 60  # hour
ES0
ER = 2.0

x = list(range(25, 50))
y = []

for B in x:
    ESe = ES0 + ER / B
    rho = labda * ESe

    # The time to form a red batch
    labda_r = 0.5
    EW_r = (B - 1) / (2 * labda_r)

    # Now the time a batch spends in queue
    Cae = 1.0
    CaB = Cae / B
    Ce = 1.0  # SCV of service times
    VS0 = Ce * ES0 * ES0
    VR = 1.0 * 1.0  # Var setups is sigma squared
    VSe = B * VS0 + VR
    ESb = B * ES0 + ER
    CeB = VSe / (ESb * ESb)
    EW = (CaB + CeB) / 2 * rho / (1 - rho) * ESb

    # The time to unpack the batch, i.e., the time at the server.
    ES = ER + (B - 1) / 2 * ES0 + ES0

    total = EW_r + EW + ES
    y.append(total)


plt.xlim(20, 50)
plt.plot(x, y, label=r"$E{J_r}$")
plt.xlabel("$B$")
# plt.ylabel("Time (hours)")
plt.title("Sojourn time of red jobs")
plt.legend(loc="lower right")
print(tikzplotlib.get_tikz_code(axis_height="5cm", axis_width="6cm"))
\end{pycode}
\end{marginfigure}

First we see a sharp decline of $\E{J_r}$.
The reason for this is that the load $\rho$ decreases as a function of $B$.
Since we know that $\E\W \sim (1-\rho)^{-1}$, it is indeed essential to stay away from critically loading the server.


When $B$ becomes quite large, we see that the sojourn time  increases linearly.
This follows right away from \cref{eq:79} and \cref{eq:85}, because the time to (dis)assemble batches is  linear in $B$.

Finally, observe that the graph is not symmetric around the minimum. It is much worse to take $B$ too small than too large.





\begin{exercise}\label{ex:103}
 Jobs\marginpar{Before dealing with technical derivations, let us first see how to apply the model.}
arrive at $\lambda=3$ per hour at a machine with $C_a^2=1$; service times are exponential with an average of 15 minutes.
Assume $\lambda_r = 0.5$ per hour, hence $\lambda_b = 2.5$ per hour.
Between any two batches, the machine requires a cleanup of 2 hours, with a standard deviation of $1$ hour.
First find  the smallest batch size that can be allowed, then compute the average time a red job spends in the system in case $B=30$ jobs.
\begin{solution}
First check the load.
\begin{pyconsole}
labda = 3  # per hour
ES0 = 15.0 / 60  # hour
ES0
ER = 2.0
Bmin = labda * ER / (1 - labda * ES0)
Bmin
\end{pyconsole}

\begin{pyconsole}
B = 30
ES = ES0 + ER / B
rho = labda * ES
rho
\end{pyconsole}

The time to form a red batch is
\begin{pyconsole}
labda_r = 0.5
EW_r = (B - 1) / (2 * labda_r)
EW_r  # in hours
\end{pyconsole}
And the time to form a blue batch is
\begin{pyconsole}
labda_b = labda - labda_r
EW_b = (B - 1) / (2 * labda_b)
EW_b  # in hours
\end{pyconsole}
The time a batch spends in queue.
\begin{pyconsole}
Cae = 1.0
CaB = Cae / B
CaB
Ce = 1.0  # SCV of service times
VS0 = Ce * ES0 * ES0
VS0
VR = 1.0 * 1.0  # Var setups is sigma squared
VS = B * VS0 + VR
VS
ESb = B * ES0 + ER
ESb
CeB = VS / (ESb * ESb)
CeB
EW = (CaB + CeB) / 2 * rho / (1 - rho) * ESb
EW
\end{pyconsole}
The time to unpack the batch, i.e., the time at the server.
\begin{pyconsole}
Eunpack = ER + (B - 1) / 2 * ES0 + ES0
Eunpack
\end{pyconsole}
The overall time red jobs spend in the system.
\begin{pyconsole}
total = EW_r + EW + Eunpack
total
\end{pyconsole}

\end{solution}
\end{exercise}



\begin{exercise}\label{ex:48}
  Show that the average time a job has to wait to fill the batch (to which this job belongs) is given by~\cref{eq:79}.
\begin{hint}
An arbitrary job has to wait half the time it takes  to form a batch.
 \end{hint}
\begin{solution}
  Suppose a batch is just finished.
  The first job of a new batch needs to wait, on average, $B-1$ inter-arrival times until the batch is complete, the second $B-2$ inter-arrival times, and so on.
  The last job does not have to wait at all.
  Thus, the total time to form a batch is $(B-1)/\lambda_r$.
  An arbitrary job can be anywhere in the batch, hence its expected time is half the total time.
\end{solution}
\end{exercise}


\begin{exercise}\label{ex:490}
Explain that the SCV of the batch inter-arrival times is given by~\cref{eq:82}.
\begin{hint}
Use  \cref{eq:95} and \cref{eq:94}.
\end{hint}
\begin{solution}
The variance of the inter-arrival time of batches is $B$ times the variance of job inter-arrival times. The inter-arrival times of batches is also $B$ times the inter-arrival times of jobs. Thus,
\begin{equation*}
 C_{a,B}^2 = \frac{B \V{X}}{(B \E X)^2} = \frac{\V X}{(\E X)^2} \frac 1 B = \frac{C_a^2}{B}.
\end{equation*}
\end{solution}
\end{exercise}


\begin{exercise}\label{ex:491}
Show that $C_{s,B}^2$ takes the form as in~\cref{eq:82}.
% \begin{hint}
%  What is the variance of a batch service time?
% \end{hint}
\begin{solution}
 The variance of a batch is $\V{R+\sum_{i=1}^B S_{0,i} } = \V R + B\V{S_0}$, since the normal service times $S_{0,i}, i=1,\ldots,B$, of the jobs are independent, and also independent of the setup time $R$ of the batch.
\end{solution}
\end{exercise}

\begin{exercise}\label{ex:492}
Show that, when items can leave right after being served, the time at the server is given by~\cref{eq:85}
\begin{solution}
 First, wait until the setup is finished, then wait (on average) for half of the batch (minus the job itself) to be served, and then the job has to be served itself, that is,
$\E{R} + \frac{B-1}{2}\E{S_0} +\E{S_0}$.
\end{solution}
\end{exercise}



\input{trailer}

%%% Local Variables:
%%% mode: latex
%%% TeX-master: t
%%% End:
