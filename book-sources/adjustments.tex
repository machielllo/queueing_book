% arara: pdflatex: { shell: yes, interaction: nonstopmode }
% arara: pythontex: {verbose: yes, rerun: modified }
% arara: pdflatex: { shell: yes, interaction: nonstopmode }
% arara: clean: { extensions: [ aux, blg, idx, ilg, ind, log, out, pytxcode, rel, toc ] }
% !arara: clean: { files: [ ans.tex, hint.tex] }

% arara: pdflatex
% arara: clean: { extensions: [ aux, blg, idx, ilg, ind, log, out, pytxcode, rel, toc ] }
% !arara: clean: { files: [ ans.tex, hint.tex] }


\documentclass[queueing-book-solution.tex]{subfiles}
\externaldocument{queueing-book}

\opt{solutionfiles,check}{
\loadgeometry{tufte}
\Opensolutionfile{hint}
\Opensolutionfile{ans}
}

\begin{document}


\section{Server Adjustments}
\label{sec:non-preempt-interr}


In~\cref{sec:setups-batch-proc} we studied the effect of setup times between batches with constant size~$B$.
However, the server can be interrupted for other reasons than setups.
For instance, parts in a machine, such as knives, may need to be replaced or adjusted.
Such small tasks can be carried out in between jobs, but it is often hard to predict when it is required.
Another example is a GP who  has to make a number of unexpected phone calls between seeing two patients.
In all such cases, the number of jobs\sidenote{You can compare this to a batch of jobs.}
served between any two adjustments\sidenote{And this to a setup.}
is no longer constant.
But we know from Sakasegawa's formula that randomness in service times affects queueing times, and we also know from~\cref{eq:60} that the batch size affects the service time.
Hence, unplanned adjustments must have an effect on sojourn times.


In this section we develop a simple model to understand the impact of such outages on job sojourn times; we use the same notation as in~\cref{sec:setups-batch-proc} and follow the same line of reasoning.
With this model, we can analyze a number of trade-offs, such doing fewer, but longer adjustments, or planning adjustments instead just waiting until it becomes necessary at an unexpected moment.\sidenote{~\cref{ex:104}} With the model we can make graphs of the sojourn time as a function of adjustment rate, so that we can optimize for the adjustment rate.

It is important to realize that setups and adjustments are both types of \recall{non-preemptive outages}, i.e., they occur \emph{between} jobs, not \emph{during}  job service times.


\newthought{We assume that} an adjustment (repair) can occur between any two jobs with some constant probability $p>0$. Let us write $F$ for the Bernouilli rv associated with such an adjustment: $F=1$ when an adjustment is necessary, and $F=0$ otherwise. Clearly,  the number of jobs served between two adjustments\sidenote{Also called a run} forms a sequence of geometric iid rvs $\{B_i\}$ such that $\P{B=k} = (1-p)^{k-1}p$.\sidenote{Recall from~\cref{eq:81} that the run-length $B$ is memoryless under these assumptions.}
As a consequence, $\E B=1/p$ and $\V B = (1-p)/p^2$.\sidenote{In practice, we measure the average number of jobs $\E B$ served between a number of adjustments, and take $p=1/\E B$.} Suppose also that the adjustment times  $\{R_{i}\}$ are iid rvs with mean $\E R$ and variance $\V R$.

\newthought{To compute $\E\W$} with Sakasegawa's formula, we only have to find out how the adjustments affect  $\E S$ and $C_s^2$.
(As the adjustments do not affect the job arrival process, $\lambda$ and $C_a^{2}$ remain the same.)

It is simple\sidenote{~\cref{ex:l-256}} to see that the average effective processing time is
\begin{equation} \label{eq:88}
 \E{S} = \E{S_0} + p \E R = \E{S_0} + \E R / \E B.
\end{equation}
As there is one server,  $\rho = \lambda \E{S}$.

With this expression for $\E S$ and an expression\sidenote{\cref{ex:78}} for $\E{S^2}$, we find\sidenote{\cref{ex:l-257}}
\begin{equation}\label{eq:89}
  \begin{split}
 \V{S}
&= \V{S_0} + p\V{R} + p (1-p)(\E R)^2 \\
&= \V{S_0} + \V R / \E B + (\E R)^2C_B^2/\E B,
 \end{split}
\end{equation}
where $C_B^2$ is the SCV of the runlength $B$.
We can now  fill in Sakasegewa's formula!


\newthought{Before considering some} examples, let us compare the results of this model to those of~\cref{sec:non-preempt-interr}.
The expected effective service time is the same: an amount $\E R/ \E B$ gets added to the net processing time $\E{S_0}$.
However, the impact on the variance is different.
By comparing~\cref{eq:107} to~\cref{eq:89} we see that the latter has an extra (positive) term.
This supports our intuition: Unexpected interruptions have a larger effect on variability than expected (planned) interruptions.


\begin{exercise}\label{ex:l-255}
 Jobs\marginpar{We first show how to apply the model before we deal with the derivations. }
arrive as a Poisson process with rate $\lambda=9$ per working day.
 The machine works two $8$ hour shifts a day.
 Work not processed on a day is carried over to the next day.
 Job service times are 1.5 hours, on average, with standard deviation of $0.5$ hours.
 Outages occur on average between $30$ jobs.
The average duration of an outage  is $5$ hours and has a standard deviation of $2$ hours.
 Compute $\E\J$.
\begin{hint}
 Get the units right.  Compute the load, and then the rest.
\end{hint}
\begin{solution}
First we determine the load.
\begin{pyconsole}
EB = 30
p = 1 / EB
ES0 = 1.5
labda = 9.0 / (2 * 8)  # arrival rate per hour
ER = 5.0
ES = ES0 + p * ER
ES
rho = labda * ES
rho
\end{pyconsole}
So, at least the system is stable.
\begin{pyconsole}
VS0 = 0.5 * 0.5
VR = 2.0 * 2.0
VS = VS0 + p * VR + p * (1 - p) * ER * ER
VS
Ce2 = VS / (ES * ES)
Ce2
\end{pyconsole}
And now we can fill in the waiting time formula.
\begin{pyconsole}
Ca2 = 1  # Poisson arrivals
EW = (Ca2 + Ce2) / 2 * rho / (1 - rho) * ES
EW
EJ = EW + ES
EJ
\end{pyconsole}
\end{solution}
\end{exercise}


\begin{exercise}\label{ex:104}
In the setting of~\cref{ex:l-255} we can perhaps choose to do an adjustment after every 20 jobs.
\marginpar{In maintenance this is a common problem. Should we wait until a failure occurs, or should we plan the repairs?}
For simplicity, assume that then adjustments never occur at random.
Also assume that an adjustment takes less time, for instance $4.5$ hour and are constant.
\marginpar{Because planned adjustment can be prepared.}
What is $\E\J$ now?
\begin{hint}
Realize that we now deal with setups and batch processing.
\end{hint}
\begin{solution}
We can use the model of~\cref{sec:setups-batch-proc}.
\begin{pyconsole}
B = 20
ER = 4.5
VR = 0
ES = ES0 + ER / B
ES
rho = labda * ES
rho
VS = VS0 + VR / B
Ce2 = VS / (ES * ES)
Ce2
EW = (Ca2 + Ce2) / 2 * rho / (1 - rho) * ES
EW
EJ = EW + ES
EJ
\end{pyconsole}
Comparing this to the results of~\cref{ex:l-255}, we see that the load becomes somewhat higher. Since $\rho$ becomes close to one, doing adjustments regularly is not a good idea.
\end{solution}
\end{exercise}


\begin{exercise}\label{ex:l-256}
 The effective processing time $S = S_{0} + R \1{F=1}$.  Use this to
derive \cref{eq:88}.
\begin{solution}
Taking expectations and using the independence of $R$ and $F$ gives the result.
 % \begin{equation*}
 % \E{S} = (1-p)\E{S_0} + p (\E{R} + \E{S_0}) = \E{S_0}  + p \E{R}.
 % \end{equation*}
\end{solution}
\end{exercise}

\begin{exercise}\label{ex:78}
Recall that $\V X = \E{X^2} - (\E X)^{2}$ for any random variable. Hence, for $\V S$ we need $\E{S^{2}}$.   Show that
 \begin{equation*}
 \E{S^2} = \E{S_0^2} + 2 p \E{S_0} \E R + p \E{R^2}.
 \end{equation*}
 \begin{hint}
 Recall that $\1{F=1}^2= \1{F=1}$. Expand the square, use that $S_0$, $R$ and $F$ are independent. Finally, $\E{\1{F=1}} = p$.
 \end{hint}
\begin{solution}

 \begin{align*}
 \E{S^2}
&= \E{(S_0+R\1{F=1})^{2}} \\
&=\E{S_0^2 + 2 S_0 R \1{F=1} + R^2\1{F=1}} \\
&=  \E{S_0^2} + 2 p \E{S_0} \E R + p \E{R^2}.
 \end{align*}
\end{solution}
\end{exercise}

\begin{exercise}\label{ex:l-257}
Derive \cref{eq:89}.
\begin{hint}
 First compute $\E{S^2}$. See~\cref{ex:78}. Don't make the error to assume that,  since $R$ and $F$ are independent,  $V{R\1{F=1}} = \V R \V{\1{F=1}} = \V R p (1-p)$.
\end{hint}
\begin{solution}
 \begin{equation*}
 \begin{split}
\V{S}
&=\E{S^2} - (\E{S})^2 \\
&= \E{S_0^2} + 2 p \E{S_0} \E R + p \E{R^2} \\
&\quad - (\E{S_0})^2 - 2p \E{S_0}\E R - (p \E R)^2\\
&= \V{S_0} + p(\E{R^2} - (\E R)^2) + (\E R)^2p (1-p) \\
&= \V{S_0} + p\V R + (\E R)^2p (1-p) \\
&= \V{S_0} + p\V R + p^3 (\E R)^2\frac{1-p}{p^2} \\
&= \V{S_0} + p\V R + p^3 (\E R)^2 \V B\\
&= \V{S_0} + p\V R + p (\E R)^2 C^2_B.
 \end{split}
 \end{equation*}
 Now replace $p$ by $1/\E B$.
\end{solution}
\end{exercise}

\input{trailer}


%%% Local Variables:
%%% mode: latex
%%% TeX-master: t
%%% End:
