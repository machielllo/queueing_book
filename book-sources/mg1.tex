% arara: pdflatex: { shell: yes, interaction: nonstopmode }
% arara: pythontex: {verbose: yes, rerun: modified }
% arara: pdflatex: { shell: yes, interaction: nonstopmode }
% arara: clean: { extensions: [ aux, blg, idx, ilg, ind, log, out, pytxcode, rel, toc ] }
% !arara: clean: { files: [ ans.tex, hint.tex] }


% arara: pdflatex
% arara: clean: { extensions: [ aux, blg, idx, ilg, ind, log, out, pytxcode, rel, toc ] }
% !arara: clean: { files: [ ans.tex, hint.tex] }


\documentclass[queueing-book-solution.tex]{subfiles}
\externaldocument{queueing-book}

\opt{solutionfiles,check}{
\loadgeometry{tufte}
\Opensolutionfile{hint}
\Opensolutionfile{ans}
}

\begin{document}



\section{$M/G/1$ Queue: Expected Waiting Time}
\label{sec:mg1}


When the service times are not really well approximated by the exponential distribution, the $M/G/1$ queue may be a more suitable model than the $M/M/1$ queue.
In this section we derive with sample path arguments the fundamentally important Pollaczek-Khinchine (PK) formula for the average waiting time in an $M/G/1$ queue.
At the  end of the section we relate Sakasegawa's formula to the PK formula.


\newthought{We start with} the simple observation that before an arriving job gets access to the server, it first has to wait until the job in service (if any) completes and then it has wait for the queue to be cleared.
From PASTA it then follows  that
$\E{\W} = \E{S_r} + \E{\Q} \E S$, where  $\E{S_r}$ is the (time-)average remaining service time of the job in service. With Little's law $\E \Q = \lambda \E\W$ and writing $\rho = \lambda \E S$, we  find that
\begin{equation*}
 \E{\W} = \frac{\E{S_r}}{1-\rho}.
\end{equation*}

\newthought{To make further} progress we need to find an expression for $\E{S_r}$ for generally distributed service times.\sidenote{For the $M/M/1$ queue, $\E{S_r} = \E S$, but not for the $M/G/1$ queue.}
For this, we can use the renewal reward theorem, just as in~\cref{sec:mxm1-queue:-expected}.

Consider the $k$th job of some sample path of the $M/G/1$ queueing process.
Let the job's service start\sidenote{Usually this is later than the arrival time $A_{k}$.} at time $\tilde A_k$, so that it departs at time $D_k=\tilde A_k + S_k$, see~\cref{fig:mg1remainingservicetime}.
At time $s$,  the remaining service time of job~$k$ is
$(D_k-s)\1{\tilde A_k \leq s < D_k}$.
\begin{marginfigure}
\begin{tikzpicture}[scale=1,
 open/.style={shape=circle, fill=white, inner sep=1pt, draw, node contents=},
 closed/.style={shape=circle, fill=black, inner sep=1pt, draw, node contents=}]
 \draw (-1,0) -- (4,0);
 %x\draw (1,0) -- (3,0) node[midway, fill=white] {$s$};
 \draw node (c1) at (0,3) [closed, label={}]
 node (c2) at (3,0)[open, label={}]
 (c1) to (c2);
 \draw[dotted] (0,0) -- (0,3) node[midway, rotate=90, fill=white] {$S_k$};
 \node[below] at (0,0) {$\tilde A_k$};
 \node[below] at (3,0) {$D_k$};

 \draw[<->, dotted] (0.75,0) -- (0.75,2.25) node[midway,fill=white,rotate=90] {$D_k-s$};
 \node[below] at (0.75,0) {$s$};
 \end{tikzpicture}

 \caption{Remaining service time.}
 \label{fig:mg1remainingservicetime}
\end{marginfigure}
By adding up all these times,  we find\sidenote{\cref{ex:13}} that
 \begin{equation*}
 Y(t) = \int_0^t (D_{D(s)+1}-s)\1{\L(s) > 0} \d s
 \end{equation*}
 is the total remaining service time as seen by the server up to time~$t$.
 Hence,   as $t \to \infty$, $Y(t)/t$ converges to the  (time-average) remaining service time $\E{S_r}$.

 We also see in \cref{fig:mg1remainingservicetime} that $X_k = Y(D_k) - Y(D_{k-1})$ is the area under the triangle.
 By choosing $T_k=D_k$ as the epochs to inspect $Y(\cdot)$, $X=\lim_{n\to\infty} \frac{1}{n}\sum_{k=1}^n S_k^2/ 2 = \E{S^2}/2$.
 By the renewal-reward theorem, $Y=\delta X$. But $\delta = \lambda$,\sidenote{by rate-stability} and therefore
\begin{equation*}
\E{S_r} = \frac{\lambda}2 \E{S^2}.
\end{equation*}

Replacing the above expression in the above expression for $\E\W$, we obtain the \recall{Pollaczek-Khinchine formula}
\begin{equation} \label{eq:710}
 \E{\W} = \frac 1 2 \frac{\lambda \E{S^2}}{1-\rho} =\frac{1 + C_s^2}2 \frac{\rho}{1-\rho} \E S,
\end{equation}
where the second equation follows after some rewriting.\sidenote{\cref{ex:65}}


\newthought{The PK formula} is an \emph{exact} result for $\E\W$, and it reduces to $\E{\W}$ for the $M/M/1$ queue by taking\sidenote{Recall, $C_s^2=1$ for the $M/M/1$ queue.}
$C_s^2=1$ in the factor $(1+C_s^2)/2$.
In a similar vein, Sakasegawa's formula for the $G/G/c$ queue reduces to the PK formula for $\E{\W}$ for the $M/G/1$ queue when the number of servers $c=1$ and setting\sidenote{Recall, $C_a^2=1$ for the $M/G/1$ queue.}
 $C_a^2=1$ in the factor $(C_a^2+C_s^2)/2$.
Again, recall that Sakasegawa's formula is an approximation, while the PK formula is exact.



\begin{exercise}\label{ex:l-241}
 Consider a workstation with just one machine.
 We model the job arrival process as a Poisson process with rate $\lambda=3$ per day.
 The average service time $\E S = 2$ hours, $C^2_s = 1/2$, and the shop is open for 8 hours per day.
Show that  $\E{\W} = 4.5h$. What would you propose to reduce $\E \W$ to 2h?
\begin{solution}
$\rho = \lambda \E S = (3/8)\cdot 2 = 3/4$, $\E{\W} = 4.5$ h.
If we were able to reduce all service variability, i.e., $C_s^2=0$, then still $\E \W = 3$h.
Hence, we have to increase capacity, or reduce $\E S$.
Another possibility is to plan the arrival of jobs such that $C_a^2=0$.
However, typically this is not possible.
Would you accept that the supermarket plans your visits?
\end{solution}
\end{exercise}




\begin{exercise}
Compare  $\E{\W(M/D/1)}$ to $\E{\W(M/M/1)}$.
\begin{hint}
Use that $\V S=0$ for the $M/D/1$ queue.
\end{hint}
\begin{solution}
$\V S = 0 \implies C_s^2 = 0 \implies \E{\W(M/D/1)} = {\E{\W(M/M/1)}}/2$.
\end{solution}
\end{exercise}

\begin{exercise}
 Compute $\E\J$ for the $M/G/1$ queue with $S\sim U[0,\alpha]$.
\begin{solution}
 \begin{align*}
\E S &= \alpha/2, & \E{S^2} &= \int_0^\alpha x^2 \d x/\alpha = \alpha^2/3,\\
\V S &= \alpha^2/3 - \alpha^2/4= \alpha^2/12, & C_s^2 &= (\alpha^2/12)/(\alpha^2/4) = 1/3,\\
\rho &= \lambda \alpha/2,\\
\E{\W} &= \frac{1+C_s^2}2 \frac{\lambda \alpha/2}{1-\lambda \alpha/2}\frac \alpha2, &
\E \J &= \E{\W} + \frac \alpha2.
 \end{align*}
\end{solution}
\end{exercise}


\begin{exercise}
Find an expression in terms of $\lambda$ and $\E S$ for the acceptance and loss probabilities of  the $M/G/1/1$ queue.
Why is this expression not valid for the $G/G/1/1$ queue.
\begin{hint}
 The rate of accepted jobs is $\lambda \pi(0)$, hence $\rho= \lambda \pi(0) \E S$.
But also $\rho = 1-\pi(0)$.  Now solve for $\pi(0)$.
\end{hint}
\begin{solution}
 \begin{equation*}
 \lambda\pi(0) \E S = 1 - \pi(0) \iff \pi(0)=\frac1{1+\lambda\E S}
\iff 1-\pi(0) = \frac{\lambda \E S}{1+\lambda \E S}.
 \end{equation*}

 There is another way to derive this.
 The system can contain at most 1 job.
 Necessarily, if the system contains a job, this job must be in service.
 All jobs that arrive while the server is busy are rejected.
 Just after a departure, the average time until the next arrival is $1/\lambda$, and then a new service starts with an average duration of $\E S$.
 After this departure, a new cycle starts.
 Thus, $\rho = \E S/(1/\lambda + \E S) = \lambda \E S/ (1 + \lambda \E S)$.


For the $G/G/1/1$ queue the PASTA property does not hold, hence $p(0)\neq \pi(0)$ in general.
\end{solution}
\end{exercise}


\begin{exercise}\label{ex:57}
 Show that for the $M/G/1$ queue, the expected idle time
 $\E I = 1/\lambda$ and the expected busy time $\E \U = \E S/(1-\rho)$.
\begin{hint}
 Inter-arrivals times  are memoryless for the $M/G/1$; for $\E U$ use the renewal-reward theorem to see that  $\rho = \E \U/(\E I + \E \U)$ for the $G/G/1$.
\end{hint}
\begin{solution}
  Right after the server becomes free, the time to a new arrival is $\sim\Exp(\lambda)$, with mean $1/\lambda$.
  For $\E U$, solve the expression of the hint with $\E I = 1/\lambda$.
  To see why $\rho = \E \U/(\E I + \E \U)$, apply the renewal reward equation.
  Take $Y(t) = \int_0^t \1{\Ls(s)=1} \d t$.
  Taking as sampling epochs the departures that leave an empty system behind, $X_k = \U_k$ and $T_k = I_k + \U_k$, where $U_k$ is the $k$th busy time, and $I_k$ the $k$th idle time.
Then $Y(t)/ t\to\rho$, $X_k/k \to \E \U$, and $k/(I_k + \U_k) \to 1/(\E I + \E U)$, which is the $\lambda$ we use in the renewal-reward theorem.

We can also obtain $\E U$ by means of a recursion. The first customer starts a busy time of average duration $\E S$. However, during this service $\lambda \E S$ new jobs arrive, in expectation. Each of these jobs restarts the busy-period. Hence, $\E U = \E S + \lambda \E S \E U$.
\end{solution}
\end{exercise}



\begin{exercise}\label{ex:13}
Explain the expression for $Y(t)$.
\begin{solution}
 At time $s$, the number of departures is $D(s)$.
 Thus, $D(s)+1$ is the first job to depart after time $s$.
 The departure time of this job is $D_{D(s)+1}$, hence the remaining service time at time $s$ is $D_{D(s)+1}-s$, provided this job is in service.
\end{solution}
\end{exercise}


\begin{exercise}\label{ex:65}
Complete the algebra in~\cref{eq:710}.
\begin{hint}
  Use~\cref{q:batch}.
\end{hint}
\begin{solution}
\begin{equation*}
 \lambda\E{S^2} = \frac{\E{S^2}}{(\E S)^2} \lambda(\E S)^2=
 \frac{\E{S^2}}{(\E S)^2} \rho \E S = (1+C_s^2) \rho \E{S}.
\end{equation*}
\end{solution}
\end{exercise}




\begin{exercise}\label{ex:5}
Show for the $M/G/1$ that $\E{S_r} = \rho \E{S_r\given S_r>0}$.
\begin{hint}
Use the PASTA property.% to Use~\cref{ex:28}.
\end{hint}
\begin{solution}
 The probability to find the server busy upon arrival is $\rho$, and $S_r>0 \iff L>0$. Therefore,
\begin{equation*}
\E{S_r} = \rho \E{S_r\given S_r >0} + (1-\rho) \E{S_r\given S_r = 0} = \rho \E{S_r\given S_r>0}.
\end{equation*}
\end{solution}
\end{exercise}

\begin{exercise}
 It is an easy mistake to think that $\E{S_r} = \E S$ when service
 times are exponential. Why is this wrong?
\begin{hint}
  Realize that when estimating $\E{S_r}$ along a sample path, $S_r=0$ for jobs that arrive at an empty system.
\end{hint}
\begin{solution}
  For the $M/M/1$, service times are memoryless, hence, $\E{S_r \given S_r>0} = \E S$.
  But, from~\cref{ex:5}, $\E{S_r} = \rho \E{S_r \given S_r>0}$ for the $M/G/1$ queue.
  The difference is due to the fact that only jobs that arrive at  a busy system see $S_r>0$.
\end{solution}
\end{exercise}



\begin{exercise}\label{ex:28}
Show for the $M/G/1$  that  with probability~$\rho$ a job leaves behind a busy server.
\begin{hint}
 Apply PASTA and \cref{eq:39}.
\end{hint}
\begin{solution}
$\rho = \sum_{i=1}^\infty p(i) = \sum_{i=1}^\infty \pi(i) = \sum_{i=1}^\infty \delta(i)$.
\end{solution}
\end{exercise}

\input{trailer}

%%% Local Variables:
%%% mode: latex
%%% TeX-master: t
%%% End:
